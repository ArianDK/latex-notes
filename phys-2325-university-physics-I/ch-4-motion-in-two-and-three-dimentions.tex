\documentclass[fleqn]{article}
\usepackage{amsmath}
\usepackage{multicol}

\usepackage[top=0.7in, bottom=0.7in, left=0.7in, right=0.7in]{geometry}

\title{Fundamentals of Physics}
\author{Arian DK}
\date{\today}

\begin{document}
\setcounter{section}{3}

\maketitle

\section{Motion in two and three dimentions}

\subsection{Position and displacment}

\textbf{Position vector} - a vector that extends from a reference point (usually the origin) to the particle.
\begin{itemize}
    \item \begin{tabular}{@{}p{6cm} l@{}}
    \textbf{Unit vector notation:} & $\vec{r} = x\hat{i} + y\hat{j} + z\hat{k}$ \\
    \end{tabular}

    \item \begin{tabular}{@{}p{6cm} l@{}}
    \textbf{Particle's displacement:} & $\Delta \vec{r} = \vec{r}_2 - \vec{r}_1$ \\
    \end{tabular}

    \item \begin{tabular}[t]{@{}p{6cm} l@{}}
    \textbf{Alternative form:} &
    $\Delta \vec{r} = (x_2 - x_1)\hat{i} + (y_2 - y_1)\hat{j} + (z_2 - z_1)\hat{k} = \Delta x\,\hat{i} + \Delta y\,\hat{j} + \Delta z\,\hat{k}$ \\
    \end{tabular}
\end{itemize}

\subsection{Average velocity and instantaneous velocity}
\begin{itemize}
    \item If a particle undergoes a displacment $\Delta\vec{r}$ in time interval $\Delta t$, it's average velocity $\vec{v}_{avg}$ for that time interval is: $\vec{v}_{avg} = \frac{\Delta \vec{r}}{\Delta t}$.
    
    \item As $\Delta t$ is shrank to 0, $\vec{v_{avg}}$ reaches a limit called either the velocity or the instantenious velocity $\vec{v}$:\\ $\vec{v} = \frac{\mathrm{d}\vec{r}}{\mathrm{d}t} = v_x\hat{i} + v_y\hat{j} + v_z\hat{k}$
\end{itemize}
$\star$ The direction of the instantaneous velocity $\vec{v}$ of a particle is always tangent to the particles path at the particles position.

\subsection{Average acceleration and instantanious acceleration}
\begin{itemize}
    \item If a particle's velocity changes from $\vec{v_1}$ to $\vec{v_2}$ in time interval $\Delta t$, it's average acceleration during $\Delta t$ is:\\ $\vec{a}_{avg} = \frac{\vec{v}_2 - \vec{v_1}}{\Delta t} = \frac{\Delta\vec{v}}{\Delta t}$
\end{itemize}

\subsection{Projectile motion}
When you throw an object at an angle:
\begin{itemize}
    \item It moves forward at a constant horizontal speed.
    \item It moves up and then down because of gravity pulling it downward.
    \item The path it follows is a parabola.
\end{itemize}
$\star$ In projectile motion,the horizontal motionand the vertical motion are independent of each other; that is, neither motion affects the other.\\
The vertical motion of a projectile is governed by the following kinematic equations:
\[
v_y = v_0 \sin \theta_0 - gt, \qquad
v_y^2 = (v_0 \sin \theta_0)^2 - 2g(y - y_0).
\]
The trajectory (path) of a particle in projectile motion is parabolic and is given by
\[
y = (\tan \theta_0)x - \frac{g x^2}{2 (v_0 \cos \theta_0)^2}, \quad \text{if } x_0 \text{ and } y_0 \text{ are zero.}
\]
The equations of motion for the particle (while in flight) can be written as
\[
x - x_0 = (v_0 \cos \theta_0)t, \qquad
y - y_0 = (v_0 \sin \theta_0)t - \tfrac{1}{2} g t^2,
\]
The particle's horizontal range $R$, which is the horizontal distance from the launch point to the point at which the particle returns to the launch height, is
\[
R = \frac{v_0^2}{g} \sin 2\theta_0.
\]
$\star$ The horizontal range $R$ is maximum for a launch angle of $45^\circ$.

\subsection{Uniform circular motion}
\[
a = \frac{v^2}{r} \quad \text{(centripetal acceleration)}, \qquad
T = \frac{2\pi r}{v} \quad \text{(period).}
\]

\subsection{Relative motion in one dimention}
In relative motion, two reference frames moving at constant velocity measure different velocities for the same particle but the same acceleration. The relationship between the measured velocities is
\[
\vec{v}_{PA} = \vec{v}_{PB} + \vec{v}_{BA},
\]
and since the frames move at constant velocity,
\[
\vec{a}_{PA} = \vec{a}_{PB}.
\]
$\star$ Observers on different frames of reference that move at constant velocity relative to each other will measure the same acceleration for a moving particle.

\subsection{Relative motion in two dimentions}
In two-dimensional relative motion, two reference frames moving at constant velocity measure different velocities for a particle but the same acceleration. The relationship between the measured velocities is
\[
\vec{v}_{PA} = \vec{v}_{PB} + \vec{v}_{BA},
\]
and since the frames move at constant velocity,
\[
\vec{a}_{PA} = \vec{a}_{PB}.
\]


\end{document}