\section{Potential energy and conservation of energy}

Types of energies and their formulas:
\begin{itemize}
    \item Potential energy: \( U(y) = mgh \)
    \item Kinetic energy: \( K = \frac{1}{2}mv^2 \)
    \item Spring energy: \( U(x) = \frac{1}{2}kx^2 \)
\end{itemize}

\subsection{Potential energy}
\noindent $\star$ The net work done by a conservative force on a particle moving around any closed path is zero.

\noindent $\star$ The work done by a conservative force on a particle moving between two points does not depend on the path taken by the particle.

\noindent $\star$ The gravitational potential energy associated with a particle–Earth system depends only on the vertical position y (or height) of the particle relative to the reference position y = 0, not on the horizontal position.

A \textit{conservative force} does zero net work around a closed path and depends only on position (e.g., gravity, springs).

The change in potential energy is related to work by
\[
\Delta U = -W = -\int_{x_i}^{x_f} F(x)\,dx.
\]

For potential energy (gravity):
\[
\Delta U = mg(y_f - y_i) = mg\Delta y, \quad \text{and if } y_i = 0, \; U(y) = mgy.
\]

For a spring (elastic potential energy):
\[
U(x) = \tfrac{1}{2}kx^2,
\]
where \( U = 0 \) when the spring is at its relaxed length.

\subsection{Conservation of mechanical energy}
\noindent $\star$ In an isolated system where only conservative forces cause energy changes, the
kinetic energy and potential energy can change, but their sum, the mechanical energy Emec of the system, cannot change.

\noindent $\star$ When the mechanical energy of a system is conserved, we can relate the sum of kinetic energy and potential energy at one instant to that at another instant without considering the intermediate motion and without finding the work done by the forces involved.


The total mechanical energy of a system is
\[
E_{\text{mec}} = K + U,
\]
where \(K\) is kinetic energy and \(U\) is potential energy.

In an isolated system (no external forces), mechanical energy is conserved:
\[
K_1 + U_1 = K_2 + U_2,
\]
or equivalently,
\[
\Delta E_{\text{mec}} = \Delta K + \Delta U = 0.
\]

\subsection{Reading a potential energy curve}
For a one-dimensional system, the force is related to the potential energy by  
\[
F(x) = -\frac{dU(x)}{dx}.
\]

The kinetic energy is  
\[
K(x) = E_{\text{mec}} - U(x),
\]
where \(E_{\text{mec}}\) is the total mechanical energy.

A \textit{turning point} occurs when \(K = 0\), and the particle reverses direction.  
An \textit{equilibrium point} occurs where \(\frac{dU(x)}{dx} = 0\), meaning \(F(x) = 0\).

\subsection{Work done on a system by an external force}
Work is energy transferred to or from a system by means of an external force acting on that system.

If no friction is involved:
\[
W = \Delta E_{\text{mec}} = \Delta K + \Delta U.
\]

When friction is present, thermal energy changes as well:
\[
W = \Delta E_{\text{mec}} + \Delta E_{\text{th}},
\]
where the thermal energy increase is
\[
\Delta E_{\text{th}} = f_k d,
\]
with \(f_k\) being the kinetic friction force and \(d\) the displacement.

\subsection{Conservation of energy}
\noindent $\star$ The total energy E of a system can change only by amounts of energy that are transferred to orfrom the system.

\noindent $\star$ The total energy E of an isolated system cannot change.

\noindent $\star$ In an isolated system, we can relate the total energy at one instant to the total energy at anotherinstant without considering the energies at intermediate times.

The total energy of a system (mechanical + thermal + internal) can only change due to work done on it:
\[
W = \Delta E = \Delta E_{\text{mec}} + \Delta E_{\text{th}} + \Delta E_{\text{int}}.
\]

For an isolated system (\(W = 0\)):
\[
\Delta E_{\text{mec}} + \Delta E_{\text{th}} + \Delta E_{\text{int}} = 0.
\]

\textbf{Power} is the rate of energy transfer.  
Average power:
\[
P_{\text{avg}} = \frac{\Delta E}{\Delta t}.
\]
Instantaneous power:
\[
P = \frac{dE}{dt}.
\]
On an energy–time graph, power equals the slope at a given point.
