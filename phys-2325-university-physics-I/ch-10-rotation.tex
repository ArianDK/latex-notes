\section{Rotation}
\subsection{Rotational variables}
\begin{itemize}
    \item To describe the rotation of a rigid body about a fixed axis, we assume a reference line fixed in the body and perpendicular to the axis of rotation. The angular position $\theta$ of this line is measured relative to a fixed direction.
    \[
        \theta = \frac{s}{r} \quad \text{(radian measure)}
    \]
    where $s$ is the arc length and $r$ is the radius of the circular path.

    \item Radian measure is related to revolutions and degrees by
    \[
        1~\text{rev} = 360^\circ = 2\pi~\text{rad}.
    \]

    \item When a body rotates from $\theta_1$ to $\theta_2$, its angular displacement is $\Delta\theta = \theta_2 - \theta_1$,
    where $\Delta\theta$ is positive for counterclockwise rotation and negative for clockwise rotation.

    \item The average angular velocity $\omega_{\text{avg}}$ is given by
    \[
        \omega_{\text{avg}} = \frac{\Delta\theta}{\Delta t},
    \]
    and the instantaneous angular velocity $\omega$ is
    \[
        \omega = \frac{d\theta}{dt}.
    \]

    \item Both $\omega_{\text{avg}}$ and $\omega$ are vectors whose directions are determined by the right-hand rule. Their magnitudes are the angular speed.

    \item If the angular velocity changes from $\omega_1$ to $\omega_2$ over a time interval $\Delta t = t_2 - t_1$, the average angular acceleration $\alpha_{\text{avg}}$ is
    \[
        \alpha_{\text{avg}} = \frac{\omega_2 - \omega_1}{t_2 - t_1} = \frac{\Delta\omega}{\Delta t},
    \]
    and the instantaneous angular acceleration $\alpha$ is
    \[
        \alpha = \frac{d\omega}{dt}.
    \]

    \item Both $\alpha_{\text{avg}}$ and $\alpha$ are vectors.
\end{itemize}

\subsection{Rotation with constant angular acceleration}
\begin{itemize}
    \item Constant angular acceleration ($\alpha = \text{constant}$) is an important special case of rotational motion. The kinematic equations are analogous to those of linear motion:

    \[
        \omega = \omega_0 + \alpha t,
    \]
    \[
        \theta - \theta_0 = \omega_0 t + \tfrac{1}{2} \alpha t^2,
    \]
    \[
        \omega^2 = \omega_0^2 + 2\alpha (\theta - \theta_0),
    \]
    \[
        \theta - \theta_0 = \tfrac{1}{2} (\omega_0 + \omega)t,
    \]
    \[
        \theta - \theta_0 = \omega t - \tfrac{1}{2} \alpha t^2.
    \]
\end{itemize}

\subsection{Relating the linear and angular variables}
\begin{itemize}
    \item A point in a rigid rotating body, at a perpendicular distance $r$ from the rotation axis, moves in a circle of radius $r$.  
    If the body rotates through an angle $\theta$, the point moves along an arc of length
    \[
        s = \theta r \quad \text{(radian measure)},
    \]
    where $\theta$ is in radians.

    \item The linear velocity $\vec{v}$ of the point is tangent to the circular path, and its magnitude (the linear speed) is
    \[
        v = \omega r \quad \text{(radian measure)},
    \]
    where $\omega$ is the angular speed (in radians per second).

    \item The linear acceleration $\vec{a}$ of the point has both tangential and radial components.  
    The tangential component is
    \[
        a_t = \alpha r \quad \text{(radian measure)},
    \]
    where $\alpha$ is the angular acceleration (in radians per second squared).  
    The radial component is
    \[
        a_r = \frac{v^2}{r} = \omega^2 r \quad \text{(radian measure)}.
    \]

    \item For uniform circular motion, the period $T$ (time for one full revolution) is given by
    \[
        T = \frac{2\pi r}{v} = \frac{2\pi}{\omega} \quad \text{(radian measure)}.
    \]
\end{itemize}

\subsection{Kinetic energy of rotation}
\subsection{Calculating the rotational of inertia}
\subsection{Torque}
\subsection{Newton's second law for rotation}
\subsection{Work and rotational kinetic energy}