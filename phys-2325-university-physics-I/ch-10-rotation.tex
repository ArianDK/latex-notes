\section{Rotation}
\subsection{Rotational variables}
\begin{itemize}
    \item To describe the rotation of a rigid body about a fixed axis, we assume a reference line fixed in the body and perpendicular to the axis of rotation. The angular position $\theta$ of this line is measured relative to a fixed direction.
    \[
        \theta = \frac{s}{r} \quad \text{(radian measure)}
    \]
    where $s$ is the arc length and $r$ is the radius of the circular path.

    \item Radian measure is related to revolutions and degrees by
    \[
        1~\text{rev} = 360^\circ = 2\pi~\text{rad}.
    \]

    \item When a body rotates from $\theta_1$ to $\theta_2$, its angular displacement is $\Delta\theta = \theta_2 - \theta_1$,
    where $\Delta\theta$ is positive for counterclockwise rotation and negative for clockwise rotation.

    \item The average angular velocity $\omega_{\text{avg}}$ is given by
    \[
        \omega_{\text{avg}} = \frac{\Delta\theta}{\Delta t},
    \]
    and the instantaneous angular velocity $\omega$ is
    \[
        \omega = \frac{d\theta}{dt}.
    \]

    \item Both $\omega_{\text{avg}}$ and $\omega$ are vectors whose directions are determined by the right-hand rule. Their magnitudes are the angular speed.

    \item If the angular velocity changes from $\omega_1$ to $\omega_2$ over a time interval $\Delta t = t_2 - t_1$, the average angular acceleration $\alpha_{\text{avg}}$ is
    \[
        \alpha_{\text{avg}} = \frac{\omega_2 - \omega_1}{t_2 - t_1} = \frac{\Delta\omega}{\Delta t},
    \]
    and the instantaneous angular acceleration $\alpha$ is
    \[
        \alpha = \frac{d\omega}{dt}.
    \]

    \item Both $\alpha_{\text{avg}}$ and $\alpha$ are vectors.
\end{itemize}
\subsection{Rotation with constant angular acceleration}

\subsection{Relating the linear and angular variables}
\subsection{Kinetic energy of rotation}
\subsection{Calculating the rotational of inertia}
\subsection{Torque}
\subsection{Newton's second law for rotation}
\subsection{Work and rotational kinetic energy}