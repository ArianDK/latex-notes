\section{Logarithmic Exponential and Other Transcedental Functions}

\subsection{The Natural Logarithmic Function: Differentiation}

\subsection*{Definition of the Natural Logarithmic Function}
The \textbf{natural logarithmic function} is defined by
\[
\ln x \;=\; \int_{1}^{x}\frac{1}{t}\,dt, 
\qquad x>0.
\]

The domain of the natural logarithmic function is the set of all positive real numbers.

\subsection*{Properties of the Natural Logarithmic Function}

The natural logarithmic function has three important properties.
\begin{enumerate}[label=\arabic*.]
  \item The domain is $(0,\infty)$ and the range is $(-\infty,\infty)$.
  \item The function is continuous, increasing, and one-to-one.
  \item The graph is concave downward.
\end{enumerate}

\subsection*{Logarithmic Properties}

If $a$ and $b$ are positive numbers and $n$ is rational, then the four properties below are true.

\begin{multicols}{2}
\begin{enumerate}[label=\arabic*.]
  \item $\ln 1 = 0$
  \item $\ln(ab) = \ln a + \ln b$
  \item $\ln(a^n) = n\ln a$
  \item $\ln\!\left(\dfrac{a}{b}\right) = \ln a - \ln b$
\end{enumerate}
\end{multicols}

\subsection*{Definition of $e$}

The letter $e$ denotes the positive real number such that
\[
\ln e \;=\; \int_{1}^{e}\frac{1}{t}\,dt \;=\; 1.
\]

\subsection*{Derivative of the Natural Logarithmic Function}

Let $u$ be a differentiable function of $x$.
\begin{enumerate}[label=\arabic*.]
  \item \[
  \frac{d}{dx}\bigl[\ln x\bigr] = \frac{1}{x},
  \qquad x>0
  \]
  \item \[
  \frac{d}{dx}\bigl[\ln u\bigr]
  = \frac{1}{u}\frac{du}{dx}
  = \frac{u'}{u},
  \qquad u>0
  \]
\end{enumerate}

\subsection*{Derivative Involving Absolute Value}

If $u$ is a differentiable function of $x$ such that $u \neq 0$, then
\[
\frac{d}{dx}\bigl[\ln|u|\bigr] = \frac{u'}{u}.
\]

\subsection{The Natural Logarithmic Function: Integration}

\subsection*{Log Rule for Integration}

Let $u$ be a differentiable function of $x$.
\begin{enumerate}[label=\arabic*.]
  \item \[
  \int \frac{1}{x}\,dx \;=\; \ln|x| + C
  \]
  \item \[
  \int \frac{1}{u}\,du \;=\; \ln|u| + C
  \]
\end{enumerate}

\subsection*{Alternative Form of the Log Rule}

Since $\dfrac{du}{dx} = u'$, we can write the Log Rule in the following useful form:
\[
\int \frac{u'}{u}\,dx \;=\; \ln|u| + C.
\]

\subsection*{Guidelines for Integration}

\begin{enumerate}[label=\arabic*.]
  \item Learn a basic list of integration formulas.
  \item Find an integration formula that resembles all or part of the integrand and, by trial and error, find a choice of $u$ that will make the integrand conform to the formula.
  \item When you cannot find a $u$-substitution that works, try altering the integrand.
  
  You might try a trigonometric identity, multiplication and division by the same quantity, addition and subtraction of the same quantity, or long division.
  \item When a graphing utility that finds antiderivatives symbolically is available, use it.
  \item Check your result by differentiating to obtain the original integrand.
\end{enumerate}

\pagebreak

\subsection*{Integrals of the Six Basic Trigonometric Functions}

\begin{multicols}{2}
\[
\int \sin u\,du \;=\; -\cos u + C
\]

\[
\int \tan u\,du \;=\; -\ln|\cos u| + C
\]

\[
\int \sec u\,du \;=\; \ln|\sec u + \tan u| + C
\]

\[
\int \cos u\,du \;=\; \sin u + C
\]

\[
\int \cot u\,du \;=\; \ln|\sin u| + C
\]

\[
\int \csc u\,du \;=\; -\ln|\csc u + \cot u| + C
\]
\end{multicols}
