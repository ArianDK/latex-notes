\section{Logarithmic Exponential and Other Transcedental Functions}

%========================================
% 5.1 The Natural Logarithmic Function: Differentiation
%========================================

\subsection{The Natural Logarithmic Function: Differentiation}

\subsection*{Definition of the Natural Logarithmic Function}
The \textbf{natural logarithmic function} is defined by
\[
\ln x \;=\; \int_{1}^{x}\frac{1}{t}\,dt, 
\qquad x>0.
\]

The domain of the natural logarithmic function is the set of all positive real numbers.

\subsection*{Properties of the Natural Logarithmic Function}

The natural logarithmic function has three important properties.
\begin{enumerate}[label=\arabic*.]
  \item The domain is $(0,\infty)$ and the range is $(-\infty,\infty)$.
  \item The function is continuous, increasing, and one-to-one.
  \item The graph is concave downward.
\end{enumerate}

\subsection*{Logarithmic Properties}

If $a$ and $b$ are positive numbers and $n$ is rational, then the four properties below are true.

\begin{multicols}{2}
\begin{itemize}[label={}, leftmargin=*]

  \item $\ln 1 = 0$

  \item $\ln(ab) = \ln a + \ln b$

  \item $\ln(a^n) = n\ln a$

  \item $\ln\!\left(\dfrac{a}{b}\right) = \ln a - \ln b$

\end{itemize}
\end{multicols}

\subsection*{Definition of $e$}

The letter $e$ denotes the positive real number such that
\[
\ln e \;=\; \int_{1}^{e}\frac{1}{t}\,dt \;=\; 1.
\]

\subsection*{Derivative of the Natural Logarithmic Function}

Let $u$ be a differentiable function of $x$.
\begin{itemize}[label={}, leftmargin=*]

  \item \[
  \frac{d}{dx}\bigl[\ln x\bigr] = \frac{1}{x},
  \qquad x>0
  \]

  \item \[
  \frac{d}{dx}\bigl[\ln u\bigr]
  = \frac{1}{u}\frac{du}{dx}
  = \frac{u'}{u},
  \qquad u>0
  \]

\end{itemize}

\subsection*{Derivative Involving Absolute Value}

If $u$ is a differentiable function of $x$ such that $u \neq 0$, then
\[
\frac{d}{dx}\bigl[\ln|u|\bigr] = \frac{u'}{u}.
\]

%========================================
% 5.2 The Natural Logarithmic Function: Integration
%========================================

\subsection{The Natural Logarithmic Function: Integration}

\subsection*{Log Rule for Integration}

Let $u$ be a differentiable function of $x$.
\begin{itemize}[label={}, leftmargin=*]

  \item \[
  \int \frac{1}{x}\,dx \;=\; \ln|x| + C
  \]

  \item \[
  \int \frac{1}{u}\,du \;=\; \ln|u| + C
  \]

\end{itemize}


\subsection*{Alternative Form of the Log Rule}

Since $\dfrac{du}{dx} = u'$, we can write the Log Rule in the following useful form:
\[
\int \frac{u'}{u}\,dx \;=\; \ln|u| + C.
\]

\subsection*{Guidelines for Integration}

\begin{enumerate}[label=\arabic*.]
  \item Learn a basic list of integration formulas.
  \item Find an integration formula that resembles all or part of the integrand and, by trial and error, find a choice of $u$ that will make the integrand conform to the formula.
  \item When you cannot find a $u$-substitution that works, try altering the integrand.
  
  You might try a trigonometric identity, multiplication and division by the same quantity, addition and subtraction of the same quantity, or long division.
  \item When a graphing utility that finds antiderivatives symbolically is available, use it.
  \item Check your result by differentiating to obtain the original integrand.
\end{enumerate}

\pagebreak

\subsection*{Integrals of the Six Basic Trigonometric Functions}

\begin{multicols}{2}
\[
\int \sin u\,du \;=\; -\cos u + C
\]

\[
\int \tan u\,du \;=\; -\ln|\cos u| + C
\]

\[
\int \sec u\,du \;=\; \ln|\sec u + \tan u| + C
\]

\[
\int \cos u\,du \;=\; \sin u + C
\]

\[
\int \cot u\,du \;=\; \ln|\sin u| + C
\]

\[
\int \csc u\,du \;=\; -\ln|\csc u + \cot u| + C
\]
\end{multicols}

%========================================
% 5.3 Inverse Functions
%========================================

\subsection{Inverse Functions}

\subsection*{Definition of Inverse Function}

A function $g$ is the \textbf{inverse function} of the function $f$ when
\[
f(g(x)) = x \quad \text{for each $x$ in the domain of $g$}
\]
and
\[
g(f(x)) = x \quad \text{for each $x$ in the domain of $f$.}
\]

The function $g$ is denoted by $f^{-1}$.

\subsection*{Reflective Property of Inverse Functions}

The graph of $f$ contains the point $(a,b)$ if and only if the graph of $f^{-1}$
contains the point $(b,a)$.

\subsection*{The Existence of an Inverse Function}

\begin{enumerate}[label=\arabic*.]
  \item A function has an inverse function if and only if it is one-to-one.
  \item If $f$ is strictly monotonic on its entire domain, then it is one-to-one and
  therefore has an inverse function.
\end{enumerate}

\subsection*{Guidelines for Finding an Inverse Function}

\begin{enumerate}[label=\arabic*.]
  \item Use Theorem 5.7 to determine whether the function $y=f(x)$ has an inverse function.
  \item Solve for $x$ as a function of $y$: $x=g(y)=f^{-1}(y)$.
  \item Interchange $x$ and $y$. The resulting equation is $y=f^{-1}(x)$.
  \item Define the domain of $f^{-1}$ as the range of $f$.
  \item Verify that $f\!\left(f^{-1}(x)\right)=x$ and $f^{-1}(f(x))=x$.
\end{enumerate}

\subsection*{Continuity and Differentiability of Inverse Functions}

Let $f$ be a function whose domain is an interval $I$. If $f$ has an inverse function,
then the following statements are true.
\begin{enumerate}[label=\arabic*.]
  \item If $f$ is continuous on its domain, then $f^{-1}$ is continuous on its domain.
  \item If $f$ is increasing on its domain, then $f^{-1}$ is increasing on its domain.
  \item If $f$ is decreasing on its domain, then $f^{-1}$ is decreasing on its domain.
  \item If $f$ is differentiable on an interval containing $c$ and $f'(c)\neq 0$, then
  $f^{-1}$ is differentiable at $f(c)$.
\end{enumerate}

\subsection*{The Derivative of an Inverse Function}

Let $f$ be a function that is differentiable on an interval $I$. If $f$ has an inverse
function $g$, then $g$ is differentiable at any $x$ for which $f'(g(x))\neq 0$.
Moreover,
\[
g'(x)=\frac{1}{f'(g(x))}, \qquad f'(g(x))\neq 0.
\]

%========================================
% 5.4 Exponential Functions: Differentiation and Integration
%========================================

\subsection{Exponential Functions: Differentiation and Integration}

\subsection*{Definition of the Natural Exponential Function}

The inverse function of the natural logarithmic function $f(x)=\ln x$ is called the
\textbf{natural exponential function} and is denoted by
\[
f^{-1}(x)=e^x.
\]

That is, $y=e^x$ if and only if $x=\ln y$.

\subsection*{Operations with Exponential Functions}

Let $a$ and $b$ be any real numbers.
\begin{itemize}[label={}, leftmargin=*]

  \item \[
  e^ae^b=e^{a+b}
  \]

  \item \[
  \frac{e^a}{e^b}=e^{a-b}
  \]

\end{itemize}

\subsection*{Properties of the Natural Exponential Function}

\begin{enumerate}[label=\arabic*.]
  \item The domain of $f(x)=e^x$ is
  \[
  (-\infty,\infty)
  \]
  and the range is
  \[
  (0,\infty).
  \]
  \item The function $f(x)=e^x$ is continuous, increasing, and one-to-one on its entire domain.
  \item The graph of $f(x)=e^x$ is concave upward on its entire domain.
  \item \[
  \lim_{x\to -\infty} e^x = 0
  \]
  \item \[
  \lim_{x\to \infty} e^x = \infty
  \]
\end{enumerate}

\subsection*{Derivatives of the Natural Exponential Function}

Let $u$ be a differentiable function of $x$.
\begin{itemize}[label={}, leftmargin=*]

  \item \[
  \frac{d}{dx}\bigl[e^x\bigr]=e^x
  \]

  \item \[
  \frac{d}{dx}\bigl[e^u\bigr]=e^u\frac{du}{dx}
  \]

\end{itemize}

\subsection*{Integration Rules for Exponential Functions}

Let $u$ be a differentiable function of $x$.
\begin{itemize}[label={}, leftmargin=*]

  \item \[
  \int e^x\,dx=e^x+C
  \]

  \item \[
  \int e^u\,du=e^u+C
  \]

\end{itemize}

%========================================
% 5.5 Bases Other than e and Applications
%========================================

\subsection{Bases Other than e and Applications}

\subsection*{Definition of Exponential Function to Base $a$}

If $a$ is a positive real number $(a\neq 1)$ and $x$ is any real number, then the
\textbf{exponential function to the base $a$} is denoted by $a^x$ and is defined by
\[
a^x = e^{(\ln a)x}.
\]

If $a=1$, then $y=1^x=1$ is a constant function.

\subsection*{Definition of Logarithmic Function to Base $a$}

If $a$ is a positive real number $(a\neq 1)$ and $x$ is any positive real number, then
the \textbf{logarithmic function to the base $a$} is denoted by $\log_a x$ and is defined as
\[
\log_a x=\frac{1}{\ln a}\ln x.
\]

\subsection*{Properties of Inverse Functions}

\begin{enumerate}[label=\arabic*.]
  \item $y=a^x$ if and only if $x=\log_a y$
  \item $a^{\log_a x}=x,$ for $x>0$
  \item $\log_a a^x=x,$ for all $x$
\end{enumerate}

\subsection*{Derivatives for Bases Other than $e$}

Let $a$ be a positive real number $(a\neq 1)$, and let $u$ be a differentiable function of $x$.
\begin{multicols}{2}
\begin{itemize}[label={}, leftmargin=*]

  \item \[
  \frac{d}{dx}\bigl[a^x\bigr]=(\ln a)a^x
  \]

  \item \[
  \frac{d}{dx}\bigl[a^u\bigr]=(\ln a)a^u\frac{du}{dx}
  \]

  \item \[
  \frac{d}{dx}\bigl[\log_a x\bigr]=\frac{1}{(\ln a)x}
  \]

  \item \[
  \frac{d}{dx}\bigl[\log_a u\bigr]=\frac{1}{(\ln a)u}\frac{du}{dx}
  \]

\end{itemize}
\end{multicols}


\subsection*{The Power Rule for Real Exponents}

Let $n$ be any real number, and let $u$ be a differentiable function of $x$.
\begin{multicols}{2}
\begin{enumerate}[label={}]
  \item \[
  \frac{d}{dx}\bigl[x^n\bigr]=nx^{n-1}
  \]
  \item \[
  \frac{d}{dx}\bigl[u^n\bigr]=nu^{n-1}\frac{du}{dx}
  \]
\end{enumerate}
\end{multicols}

\subsection*{A Limit Involving $e$}

\[
\lim_{x\to\infty}\left(1+\frac{1}{x}\right)^x
=\lim_{x\to\infty}\left(\frac{x+1}{x}\right)^x
=e
\]

\pagebreak

\subsection*{Summary of Compound Interest Formulas}

In the formulas below, $P$ is the amount deposited, $t$ is the number of years, $A$ is
the balance after $t$ years, $r$ is the annual interest rate (in decimal form), and $n$
is the number of compoundings per year.
\begin{enumerate}[label=\arabic*.]
  \item Compounded $n$ times per year:
  \[
  A=P\left(1+\frac{r}{n}\right)^{nt}
  \]
  \item Compounded continuously:
  \[
  A=Pe^{rt}
  \]
\end{enumerate}

%========================================
% 5.6 Indeterminate Forms and L'Hopital's Rule
%========================================
\subsection{Indeterminate Forms and L'H\^{o}pital's Rule}

\subsection*{The Extended Mean Value Theorem}

If $f$ and $g$ are differentiable on an open interval $(a,b)$ and continuous on $[a,b]$
such that $g'(x)\neq 0$ for any $x$ in $(a,b)$, then there exists a point $c$ in $(a,b)$
such that
\[
\frac{f'(c)}{g'(c)}=\frac{f(b)-f(a)}{g(b)-g(a)}.
\]

\subsection*{L'H\^{o}pital's Rule}

Let $f$ and $g$ be functions that are differentiable on an open interval $(a,b)$ containing
$c$, except possibly at $c$ itself. Assume that $g'(x)\neq 0$ for all $x$ in $(a,b)$, except
possibly at $c$ itself. If the limit of $f(x)/g(x)$ as $x$ approaches $c$ produces the
indeterminate form $0/0$, then
\[
\lim_{x\to c}\frac{f(x)}{g(x)}=\lim_{x\to c}\frac{f'(x)}{g'(x)}
\]
provided the limit on the right exists (or is infinite). This result also applies when the
limit of $f(x)/g(x)$ as $x$ approaches $c$ produces any one of the indeterminate forms
$\infty/\infty$, $(-\infty)/\infty$, $\infty/(-\infty)$, or $(-\infty)/(-\infty)$.

%========================================
% 5.7 Inverse Trigonometric Functions: Differentiation
%========================================
\subsection{Inverse Trigonometric Functions: Differentiation}

\subsection*{Definitions of Inverse Trigonometric Functions}

\begin{center}
\renewcommand{\arraystretch}{1.6}
\begin{tabular}{p{2.9in} p{1.6in} p{2.2in}}
\textbf{Function} & \textbf{Domain} & \textbf{Range} \\[4pt]
$y=\arcsin x \ \text{iff}\ \sin y=x$
& $-1\le x\le 1$
& $-\dfrac{\pi}{2}\le y\le \dfrac{\pi}{2}$ \\

$y=\arccos x \ \text{iff}\ \cos y=x$
& $-1\le x\le 1$
& $0\le y\le \pi$ \\

$y=\arctan x \ \text{iff}\ \tan y=x$
& $-\infty<x<\infty$
& $-\dfrac{\pi}{2}<y<\dfrac{\pi}{2}$ \\

$y=\arccot x \ \text{iff}\ \cot y=x$
& $-\infty<x<\infty$
& $0<y<\pi$ \\

$y=\arcsec x \ \text{iff}\ \sec y=x$
& $|x|\ge 1$
& $0\le y\le \pi,\quad y\neq \dfrac{\pi}{2}$ \\

$y=\arccsc x \ \text{iff}\ \csc y=x$
& $|x|\ge 1$
& $-\dfrac{\pi}{2}\le y\le \dfrac{\pi}{2},\quad y\neq 0$
\end{tabular}
\end{center}

\subsection*{Properties of Inverse Trigonometric Functions}

If $-1\le x\le 1$ and $-\pi/2\le y\le \pi/2$, then
\[
\sin(\arcsin x)=x
\qquad \text{and} \qquad
\arcsin(\sin y)=y.
\]
If $-\pi/2<y<\pi/2$, then
\[
\tan(\arctan x)=x
\qquad \text{and} \qquad
\arctan(\tan y)=y.
\]
If $|x|\ge 1$ and $0\le y<\pi/2$ or $\pi/2<y\le \pi$, then
\[
\sec(\arcsec x)=x
\qquad \text{and} \qquad
\arcsec(\sec y)=y.
\]

Similar properties hold for the other inverse trigonometric functions.

\subsection*{Derivatives of Inverse Trigonometric Functions}

Let $u$ be a differentiable function of $x$.
\begin{multicols}{2}
\begin{itemize}[label={}, leftmargin=*]

  \item \[
  \frac{d}{dx}\bigl[\arcsin u\bigr]
  =\frac{u'}{\sqrt{1-u^2}}
  \]

  \item \[
  \frac{d}{dx}\bigl[\arccos u\bigr]
  =\frac{-u'}{\sqrt{1-u^2}}
  \]

  \item \[
  \frac{d}{dx}\bigl[\arctan u\bigr]
  =\frac{u'}{1+u^2}
  \]

  \item \[
  \frac{d}{dx}\bigl[\arccot u\bigr]
  =\frac{-u'}{1+u^2}
  \]

  \item \[
  \frac{d}{dx}\bigl[\arcsec u\bigr]
  =\frac{u'}{|u|\sqrt{u^2-1}}
  \]

  \item \[
  \frac{d}{dx}\bigl[\arccsc u\bigr]
  =\frac{-u'}{|u|\sqrt{u^2-1}}
  \]

\end{itemize}
\end{multicols}

\pagebreak

\subsection*{Basic Differentiation Rules for Elementary Functions}

\begin{center}
\begin{multicols}{2}
\setlength{\parskip}{0pt}

\begin{itemize}[label={}, leftmargin=*]
\setlength{\itemsep}{2pt}

  \item $\displaystyle \frac{d}{dx}[cu]=cu'$

  \item $\displaystyle \frac{d}{dx}[u\pm v]=u'\pm v'$

  \item $\displaystyle \frac{d}{dx}[uv]=uv'+vu'$

  \item $\displaystyle \frac{d}{dx}\left[\frac{u}{v}\right]=\frac{vu'-uv'}{v^2}$

  \item $\displaystyle \frac{d}{dx}[c]=0$

  \item $\displaystyle \frac{d}{dx}[u^n]=nu^{\,n-1}u'$

  \item $\displaystyle \frac{d}{dx}[x]=1$

  \item $\displaystyle \frac{d}{dx}[|u|]=\frac{u}{|u|}(u'), \quad u\neq 0$

  \item $\displaystyle \frac{d}{dx}[\ln u]=\frac{u'}{u}$

  \item $\displaystyle \frac{d}{dx}[e^u]=e^u u'$

  \item $\displaystyle \frac{d}{dx}[\log_a u]=\frac{u'}{(\ln a)u}$

  \item $\displaystyle \frac{d}{dx}[a^u]=(\ln a)a^u u'$

  \item $\displaystyle \frac{d}{dx}[\sin u]=(\cos u)u'$

  \item $\displaystyle \frac{d}{dx}[\cos u]=-(\sin u)u'$

  \item $\displaystyle \frac{d}{dx}[\tan u]=(\sec^2 u)u'$

  \item $\displaystyle \frac{d}{dx}[\cot u]=-(\csc^2 u)u'$

  \item $\displaystyle \frac{d}{dx}[\sec u]=(\sec u\tan u)u'$

  \item $\displaystyle \frac{d}{dx}[\csc u]=-(\csc u\cot u)u'$

  \item $\displaystyle \frac{d}{dx}[\arcsin u]=\frac{u'}{\sqrt{1-u^2}}$

  \item $\displaystyle \frac{d}{dx}[\arccos u]=\frac{-u'}{\sqrt{1-u^2}}$

  \item $\displaystyle \frac{d}{dx}[\arctan u]=\frac{u'}{1+u^2}$

  \item $\displaystyle \frac{d}{dx}[\arccot u]=\frac{-u'}{1+u^2}$

  \item $\displaystyle \frac{d}{dx}[\arcsec u]=\frac{u'}{|u|\sqrt{u^2-1}}$

  \item $\displaystyle \frac{d}{dx}[\arccsc u]=\frac{-u'}{|u|\sqrt{u^2-1}}$

\end{itemize}
\end{multicols}
\end{center}




%========================================
% 5.8 Inverse Trigonometric Functions: Integration
%========================================

\subsection{Inverse Trigonometric Functions: Integration}

\subsection*{Integrals Involving Inverse Trigonometric Functions}

Let $u$ be a differentiable function of $x$, and let $a>0$.
\begin{center}
\setlength{\parskip}{2pt}

$\displaystyle \int \frac{du}{\sqrt{a^2-u^2}}
=\arcsin\frac{u}{a}+C$

$\displaystyle \int \frac{du}{a^2+u^2}
=\frac{1}{a}\arctan\frac{u}{a}+C$

$\displaystyle \int \frac{du}{u\sqrt{u^2-a^2}}
=\frac{1}{a}\arcsec\frac{|u|}{a}+C$

\end{center}

\pagebreak

\subsection*{Basic Integration Rules ($a>0$)}

\begin{multicols}{2}
\setlength{\parskip}{0pt}

\begin{itemize}[label={}, leftmargin=*]
\setlength{\itemsep}{2pt}

  \item $\displaystyle \int kf(u)\,du = k\int f(u)\,du$

  \item $\displaystyle \int [f(u)\pm g(u)]\,du
  = \int f(u)\,du \pm \int g(u)\,du$

  \item $\displaystyle \int du = u + C$

  \item $\displaystyle \int u^n\,du = \frac{u^{n+1}}{n+1}+C,
  \quad n\neq -1$

  \item $\displaystyle \int \frac{du}{u} = \ln|u| + C$

  \item $\displaystyle \int e^u\,du = e^u + C$

  \item $\displaystyle \int a^u\,du = \left(\frac{1}{\ln a}\right)a^u + C$

  \item $\displaystyle \int \sin u\,du = -\cos u + C$

  \item $\displaystyle \int \cos u\,du = \sin u + C$

  \item $\displaystyle \int \tan u\,du = -\ln|\cos u| + C$

  \item $\displaystyle \int \cot u\,du = \ln|\sin u| + C$

  \item $\displaystyle \int \sec u\,du = \ln|\sec u+\tan u| + C$

  \item $\displaystyle \int \csc u\,du = -\ln|\csc u+\cot u| + C$

  \item $\displaystyle \int \sec^2 u\,du = \tan u + C$

  \item $\displaystyle \int \csc^2 u\,du = -\cot u + C$

  \item $\displaystyle \int \sec u\tan u\,du = \sec u + C$

  \item $\displaystyle \int \csc u\cot u\,du = -\csc u + C$

  \item $\displaystyle \int \frac{du}{\sqrt{a^2-u^2}}
  = \arcsin\frac{u}{a} + C$

  \item $\displaystyle \int \frac{du}{a^2+u^2}
  = \frac{1}{a}\arctan\frac{u}{a} + C$

  \item $\displaystyle \int \frac{du}{u\sqrt{u^2-a^2}}
  = \frac{1}{a}\arcsec\frac{|u|}{a} + C$

\end{itemize}
\end{multicols}


%========================================
% 5.8 Hyperbolic Functions
%========================================
\subsection{Hyperbolic Functions}

\subsection*{Definitions of the Hyperbolic Functions}

\begin{multicols}{2}
\[
\sinh x=\frac{e^x-e^{-x}}{2}
\]

\[
\cosh x=\frac{e^x+e^{-x}}{2}
\]

\[
\tanh x=\frac{\sinh x}{\cosh x}
\]

\[
\csch x=\frac{1}{\sinh x}, \quad x\neq 0
\]

\[
\sech x=\frac{1}{\cosh x}
\]

\[
\coth x=\frac{1}{\tanh x}, \quad x\neq 0
\]
\end{multicols}

\pagebreak

\subsection*{Hyperbolic Identities}

\begin{multicols}{2}
\[
\cosh^2x-\sinh^2x=1
\]

\[
\tanh^2x+\sech^2x=1
\]

\[
\coth^2x-\csch^2x=1
\]

\[
\sinh^2x=\frac{-1+\cosh 2x}{2}
\]

\[
\sinh 2x=2\sinh x\cosh x
\]

\[
\sinh(x+y)=\sinh x\cosh y+\cosh x\sinh y
\]

\[
\sinh(x-y)=\sinh x\cosh y-\cosh x\sinh y
\]

\[
\cosh(x+y)=\cosh x\cosh y+\sinh x\sinh y
\]

\[
\cosh(x-y)=\cosh x\cosh y-\sinh x\sinh y
\]

\[
\cosh^2x=\frac{1+\cosh 2x}{2}
\]

\[
\cosh 2x=\cosh^2x+\sinh^2x
\]
\end{multicols}

\subsection*{Derivatives and Integrals of Hyperbolic Functions}

Let $u$ be a differentiable function of $x$.
\begin{multicols}{2}
\[
\frac{d}{dx}[\sinh u]=(\cosh u)u'
\]

\[
\frac{d}{dx}[\cosh u]=(\sinh u)u'
\]

\[
\frac{d}{dx}[\tanh u]=(\sech^2u)u'
\]

\[
\frac{d}{dx}[\coth u]=-(\csch^2u)u'
\]

\[
\frac{d}{dx}[\sech u]=-(\sech u\,\tanh u)u'
\]

\[
\frac{d}{dx}[\csch u]=-(\csch u\,\coth u)u'
\]

\[
\int \cosh u\,du=\sinh u+C
\]

\[
\int \sinh u\,du=\cosh u+C
\]

\[
\int \sech^2u\,du=\tanh u+C
\]

\[
\int \csch^2u\,du=-\coth u+C
\]

\[
\int \sech u\,\tanh u\,du=-\sech u+C
\]

\[
\int \csch u\,\coth u\,du=-\csch u+C
\]
\end{multicols}

\subsection*{Inverse Hyperbolic Functions}

\begin{center}
\renewcommand{\arraystretch}{1.7}
\begin{tabular}{p{3.6in} p{2.0in}}
\textbf{Function} & \textbf{Domain} \\[4pt]
$\sinh^{-1}x=\ln\!\left(x+\sqrt{x^2+1}\right)$
& $(-\infty,\infty)$ \\

$\cosh^{-1}x=\ln\!\left(x+\sqrt{x^2-1}\right)$
& $[1,\infty)$ \\

$\tanh^{-1}x=\dfrac{1}{2}\ln\!\left(\dfrac{1+x}{1-x}\right)$
& $(-1,1)$ \\

$\coth^{-1}x=\dfrac{1}{2}\ln\!\left(\dfrac{x+1}{x-1}\right)$
& $(-\infty,-1)\cup(1,\infty)$ \\

$\sech^{-1}x=\ln\!\left(\dfrac{1+\sqrt{1-x^2}}{x}\right)$
& $(0,1]$ \\

$\csch^{-1}x=\ln\!\left(\dfrac{1}{x}+\dfrac{\sqrt{1+x^2}}{|x|}\right)$
& $(-\infty,0)\cup(0,\infty)$
\end{tabular}
\end{center}

\subsection*{Differentiation and Integration Involving Inverse Hyperbolic Functions}

Let $u$ be a differentiable function of $x$.
\begin{multicols}{2}
\[
\frac{d}{dx}[\sinh^{-1}u]=\frac{u'}{\sqrt{u^2+1}}
\]

\[
\frac{d}{dx}[\cosh^{-1}u]=\frac{u'}{\sqrt{u^2-1}}
\]

\[
\frac{d}{dx}[\tanh^{-1}u]=\frac{u'}{1-u^2}
\]

\[
\frac{d}{dx}[\coth^{-1}u]=\frac{u'}{1-u^2}
\]

\[
\frac{d}{dx}[\sech^{-1}u]=\frac{-u'}{u\sqrt{1-u^2}}
\]

\[
\frac{d}{dx}[\csch^{-1}u]=\frac{-u'}{|u|\sqrt{1+u^2}}
\]
\end{multicols}

\[
\int \frac{du}{\sqrt{u^2\pm a^2}}
=\ln\!\left(u+\sqrt{u^2\pm a^2}\right)+C
\]

\[
\int \frac{du}{a^2-u^2}
=\frac{1}{2a}\ln\left|\frac{a+u}{a-u}\right|+C
\]

\[
\int \frac{du}{u\sqrt{a^2\pm u^2}}
=-\frac{1}{a}\ln\!\left(\frac{a+\sqrt{a^2\pm u^2}}{|u|}\right)+C
\]
