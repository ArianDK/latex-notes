\section{Logarithmic Exponential and Other Transcedental Functions}

\subsection{The Natural Logarithmic Function: Differentiation}

\subsection*{Definition of the Natural Logarithmic Function}
The \textbf{natural logarithmic function} is defined by (fill in the function and bound)
\[
\ln x \;=\; \int_{1}^{x}\frac{\,\,\,\,}{\,\,\,\,}\,dt, 
\qquad x>\,\,\,\,.
\]

The domain of the natural logarithmic function is the set of all positive real numbers.

\subsection*{Properties of the Natural Logarithmic Function}

The natural logarithmic function has three important properties (cross out the word that is incorrect).
\begin{enumerate}[label=\arabic*.]
  \item The domain is $(\,\,\,\,,\,\,\,\,)$ and the range is $(\,\,\,\,,\,\,\,\,)$.
  \item The function is discrete/continuous, decreasing/increasing, and one-to-one.
  \item The graph is convex/concave downward.
\end{enumerate}

\subsection*{Logarithmic Properties}

If $a$ and $b$ are positive numbers and $n$ is rational, then the four properties below are true.

\begin{multicols}{2}
\begin{enumerate}[label=\arabic*.]
  \item $\ln 1 =$
  \item $\ln(ab) =$
  \item $\ln(a^n) =$
  \item $\ln\!\left(\dfrac{a}{b}\right) =$
\end{enumerate}
\end{multicols}

\subsection*{Definition of $e$}

The letter $e$ denotes the positive real number such that
\[
\ln e \;=\; \int_{1}^{e}\frac{1}{t}\,dt \;=\; \,\,\,\,.
\]

\subsection*{Derivative of the Natural Logarithmic Function}

Let $u$ be a differentiable function of $x$.
\begin{enumerate}[label=\arabic*.]
  \item \[
  \frac{d}{dx}\bigl[\ln x\bigr] = \frac{\,\,\,\,}{\,\,\,\,},
  \qquad x>0
  \]
  \item \[
  \frac{d}{dx}\bigl[\ln u\bigr]
  = \frac{\,\,\,\,}{\,\,\,\,}\frac{du}{dx}
  = \frac{\,\,\,\,}{\,\,\,\,},
  \qquad u>0
  \]
\end{enumerate}

\subsection*{Derivative Involving Absolute Value}

If $u$ is a differentiable function of $x$ such that $u \neq 0$, then
\[
\frac{d}{dx}\bigl[\ln|u|\bigr] = \frac{u'}{u}.
\]
